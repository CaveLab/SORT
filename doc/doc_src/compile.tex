\documentclass[10pt,a4paper]{article}

\usepackage[hidelinks=true]{hyperref}
\usepackage{color}

\title{ To Compile SORT }
\author{ Jiayin Cao }

\begin{document}

\maketitle

\section{ Brief Introduction }
SORT is short for Simple Open-source Ray Tracing.
It is a open source project on 
\href{http://sourceforge.net/projects/soraytrace/?source=directory}{\color{blue}{sourceforge}}.
Any one is free to check out the source code.

SORT is a multi-platforms project which could work on Windows , Ubuntu and Mac OS.
There are varies way to compile the project.

\section{ To compile on different platforms }
\subsection{ Compile on Windows }
SORT is mainly developed on Windows so that to compile it on the platform is quite easy.
Visual studio 2008 is required to compile the source code.
Other version of visual studio could work too, but you will have to create your own .sln file.

There is a directory named "windows" on the root directory of SORT.
The sln file is in the directory.
Open it , build , it's done.

\subsection{ Compile on Mac OS }
Mac Os comiling is the last supported one.
There are two ways of doing it.
\subsubsection{ Use Makefile }
To compile this way G++ is required. And G++ could be set by installing Xcode (Xcode command line tools , which is not installed by default, is required too.)
There is a makefile in "src" directory. 
A simple "make" command on the terminal will do.

The advantage of this compiling is that it is way simpler than the others,
while the bad thing about it is that it is very hard to debug the program if you are not familiar with vi and gdb.

\subsubsection{ Use Xcode }
Xcode is required to compile this way.
In my case, Xcode 4.3.3 is used which means that Mac Os lion is required too.
It is simple to compile using xcode too.
There is a directory "mac" in the root directory of SORT.
And the project file is there.

It is easy to compile too.
And you can debug the program if you need to.
Xcode provides us with rich features helping debugging.
Whether you are familiar with vi or gdb is not an issue.

\subsection{ Compile on Ubuntu }
There is nothing much to say about it.
It is the same way with compiling using "makefile" on Mac OS.

\section{ To run on different platforms }
Whichever way of compiling you choose, the executable file is generated in a new directory "bin".
To run the program is nothing but execute it.
A log file will be generated for debug purpose.

\section{ About 3rd-party library }
It's easy to compile and run on each of the different platforms.
All the third-party library is open-sourced, all of which are included in the project.
So there is no need to download them seperately at all which makes the compiling so easy.
Note, there is memory leak in some of the open-source library and it's all fixed which means the third-party is sort of a modified version you can't download anywhere.

\end{document}